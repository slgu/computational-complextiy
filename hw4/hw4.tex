\documentclass[11pt]{article}
\usepackage{enumerate}
\usepackage[T1]{fontenc}
\usepackage[top=1in, bottom=1in, left=1.25in, right=1.25in]{geometry}
\usepackage{graphicx}
\newcommand{\itab}[1]{\hspace{0em}\rlap{#1}}
\newcommand{\tab}[1]{\hspace{1em}\rlap{#1}}
\title{Homework 4 and 5 for Intro to Computational Complextiy}
\author{Shenlong Gu}
\date{917-544-8927, sg3301@columbia.edu, 25, Nov 2015}
\begin{document}
\maketitle
\part{}
Let's assume NP $\subseteq$ BPP. We will prove SAT which is a NP-Complete problem $\in$ RP to solve this problem. We know we have a BPP machine to decide language in some probability. First we can know given a string $x$, with length $n$, we can run BPP machine several times and use the major vote and judge if $l$ is $\in$ $SAT$ with $p$ = $\frac{1}{2 * n}$. And we can use this property to construct a RP machine to decide $SAT$. Let m-BPP to be the modified BPP (running input $l$ serveral times $r(n)$ (poly-nomial with input length $n$ to get the answer) \\
First, we use m-BPP to judge if $l$ belongs to SAT, if no, rej. Then use m-BPP to judge if set $x_{0}$ to be 0 and if $l^{'}$ belongs to SAT, if no, we set $x_{0}$ to be 1 and continue to set $x_{1}, x_{2}, ...x{n}$. Finally, we got an assignment for each $x_{i}$, we judge if this assignment satisfied $l$, if yes, accept, if no, rej. \\ 
It is easy to see, if the $l$ $\notin$ SAT, we won't get a satisfiable assignment and this machine will always rej. If $l$ $\in$ SAT, we can calculate the possiblity, for n steps that m-BPP makes the wrong decision. The probability m-BPP making some wrong decision in total is at most $n$ * $\frac{1}{2 * n}$ = $\frac{1}{2}$. So we prove SAT $\in$ RP. so NP $\subseteq$ RP. And we know that RP $\subseteq$ NP, so NP = RP.
\part{}
First, it is easy to show Oracle \#P is more powerful than Oracle PP, because if we know the count of all accepting path for a NTM, it is easy to calculate $Pr[D(x,y)]$ and is easy to judge whether to accept or reject the input and PP class can just tell the most significant bit in the \#P problem answer. So $P^{PP}$ $\subseteq$ $P^{\#P}$, now we will prove another side, $P^{\#P}$ $\subseteq$ $P^{PP}$. 
\part{}
Because we know $\xi_{2}$ is larger than $\xi_{1}$, we just use Chernoff bound for the middle point $\frac{\xi_{1} + \xi_{2}}{2}$.
We will run a random turning machine for D for $n$ times, and compare the total sum of the result to $\frac{\xi_{1} + \xi_{2}}{2}$ * $n$. if the total sum is larger than it, we accept, otherwise we will reject. We will use two useful forms of Chernoff bound to prove this. \\
Given $n$ independent random varibles, $x_{1}$..$x_{n}$ between [0, 1], let $X$ to be the sum of them and $u$ to be the expected value of $X$. \\
Pr(X $\leq$ (1 - $\sigma$) * $u$) $\leq$ $e^{\frac{-\sigma^{2}*u}{2}}$ \\
Pr(X $\geq$ (1 + $\sigma$) * $u$) $\leq$ $e^{\frac{-\sigma^{2}*u}{3}}$ \\
We will find that regardless of how many times we run this randomized algorithm, the $\sigma$ for these two formulas won't change. For the first prove, we set $\sigma$ to be $\frac{\xi_{2} - \xi_{1}}{2*\xi_{2}}$, and for the second prove, we set $\sigma$ to be $\frac{\xi_{2} - \xi_{1}}{2*\xi_{1}}$. And we can just adjust the $n$ to run the randomized algorithms for many times, and make $e^{\frac{-\sigma^{2}*u}{2}}$ and $e^{\frac{-\sigma^{2}*u}{3}}$ to be less than $\frac{1}{3}$. So we have proved that $L$ belongs to BPP.
\part{}
Assume SAT belongs to P/log(n). We will begin by using the self-reducibility of SAT. We give an algorithm for SATPath problem which helps finding a path for a circuit, returning None if the circuit is not satisfiable.\\
First if $\varphi$ $\notin$ SAT, returns None. \\
For i in (1,n): \\
	we just check if $\varphi$ with $x_{1}..x_{i-1}$ set and $x_{i}$ set to be 0 $\in$ SAT. If no, set $x_{i}$ = 1. \\
Return the assignment for $x_{1}..x_{i}$. \\
From this algorithm, we can find the SATPath problem $\in$ P/log(n), then we will give a poly-nomial algorithm for SAT, we just loop every possible advice string, (just O(n) possibilities), and run the TM for SATPath, we check each answer assignment (if the result is not none), if the assigment trully satisfies the formula, accept. After loop for all possible advice string, if we accept no result, we reject.
\part{}
Assume we know PSPACE $\subseteq$ P/poly, and because P/poly is the same as a series of poly-size circuits, we will prove a PSPACE-Complete problem, TQBF $\in$ $\Sigma_{2}$. Given a TQBF formula with size $m$, we know there is a series of poly-size circuits, $C_{1}..C_{m}$ to solves TQBF game with less than size $m$. So what we will do is to guess these circuits and check the correctness of these circuits. The algorithm in $\Sigma_{2}$ is below:
Given a TQBF formula $\Phi$ with m variables. \\
For the $\exists$ statement, we randomly guess a series of poly-size circuits first, $C_{1}..C_{m}$. \\
For the $\forall$ statement, we randomly check a sub-TQBF game $\Phi^{'}$ with $x_{1}..x_{i}$ set, if the quantifier is an existential quantifier, we just check if the output of $C_{k}$ for $\Phi^{'}$ is 
the output of $C_{k-1}$ for $\Phi^{''}$ with $x_{i+1}$ set to be 1 || the output of $C_{k-1}$ for $\Phi^{'''}$ with $x_{i+1}$ set to be 0. If the quantifier is an universal quantifier, we just check if the output of $C_{k}$ for $\Phi^{'}$ is the output of $C_{k-1}$ for $\Phi^{''}$ with $x_{i+1}$ set to be 1 \&\& the output of $C_{k-1}$ for $\Phi^{'''}$ with $x_{i+1}$ set to be 0. \\
So we have proved that Given PSPACE $\subseteq$ P/poly, PSPACE = $\Sigma_{2}$
\part{}
From calculating number of perfect matching in the bipartite graph, we know 0-1 permanent of matrix counting problem is \#P-complete.
\part{}
Assume size of S is $n$. \\
From the problem, we know H(s) != t means there is a t for every s, H(s) != t, because h is from an universal hash family, so given a specific t, 
\part{}
We can use the similar approach to give a randomized algorithm to approximate \#$\phi$ as told in class. Assume $2^{k}$ $\leq$ |S| $\leq$ $2^{k+1}$. We just want to get the $k$ to use $2^{k+1}$ to approximate the answer. We consider a problem, given a set $T$ $\{0,1\}^{m}$, we randomly select $r$ hash function $h_{i}$ from the universal hash family. Consider a decision problem, if one of these hash functions make h(S) = T. We can draw a chart, size of $T$ from 0 to $m$. It is easy to show, when $m$ is too small, we use the inequality of question 2, when 
$\frac{2^{2m+1}}{|S|}$ is less than $\frac{1}{2}$, use a larger $r$ almost the decision problem provided will return false, and as $r$ get larger, the decision problem provided will return true for some probability because we randomly select hash function $h_{1}...h_{r}$, and we can similarly use SAT-Oracle to solve this decision problem, and consider the use this decision problem, if $m$ is set to be $\geq$ $k+1$, we can not get:  h(S) = T, and use $2*m+1$ $\leq$ $k-1$, we can know the probability Pr(H(S) = T) $\geq$ 1 - $\frac{2^{2m+1}}{|S|}$ $\geq$ $\frac{1}{2}$. So pr that one of the hash function satisfiy H(S) = T is $\geq$ 1 - $\frac{1}{2^{r}}$, almost 1. \\
By far we can know if $m$ less than $\frac{k}{2}$, decision problem will always output yes, is $m$ is larger than $k$, the decision problem will always output no, the undefine region is between $k/2$ to $k$, so we use the first $m$ that outputs yes, and use $M^{*}$ = $2^{m}$ to be the answer. we know $\frac{M^{*}}{2^(k/2)}$ $\leq$ \#$\phi$ $\leq$ $M^{*}$, and we will apply the similar tricky approach to this boundary, we just put n $\phi$ with length n together to get a $\phi^{'}$ with length $n^{2}$, and get the approach $M^{*}$, then use $\sqrt[n]{M^{*}}$ to be the approximating answer. And we consider the divisor will be $\sqrt[n]{2^{k/2}}$ $\leq$ 2, so we get the right approximating answer.
\end{document} 