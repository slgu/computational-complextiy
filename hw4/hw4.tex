\documentclass[11pt]{article}
\usepackage{enumerate}
\usepackage[T1]{fontenc}
\usepackage[top=1in, bottom=1in, left=1.25in, right=1.25in]{geometry}
\usepackage{graphicx}
\newcommand{\itab}[1]{\hspace{0em}\rlap{#1}}
\newcommand{\tab}[1]{\hspace{1em}\rlap{#1}}
\title{Homework 4 and 5 for Intro to Computational Complextiy}
\author{Shenlong Gu}
\date{917-544-8927, sg3301@columbia.edu, 25, Nov 2015}
\begin{document}
\maketitle
\part{}
Let's assume NP $\subseteq$ BPP. We will prove SAT which is a NP-Complete problem $\in$ RP to solve this problem. We know we have a BPP machine to decide language in some probability. First we can know given a string $x$, with length $n$, we can run BPP machine several times and use the major vote and judge if $l$ is $\in$ $SAT$ with $p$ = $\frac{1}{2 * n}$. And we can use this property to construct a RP machine to decide $SAT$. Let m-BPP to be the modified BPP (running input $l$ serveral times $r(n)$ (poly-nomial with input length $n$ to get the answer) \\
First, we use m-BPP to judge if $l$ belongs to SAT, if no, rej. Then use m-BPP to judge if set $x_{0}$ to be 0 and if $l^{'}$ belongs to SAT, if no, we set $x_{0}$ to be 1 and continue to set $x_{1}, x_{2}, ...x{n}$. Finally, we got an assignment for each $x_{i}$, we judge if this assignment satisfied $l$, if yes, accept, if no, rej. \\ 
It is easy to see, if the $l$ $\notin$ SAT, we won't get a satisfiable assignment and this machine will always rej. If $l$ $\in$ SAT, we can calculate the possiblity, for n steps that m-BPP makes the wrong decision. The probability m-BPP making some wrong decision in total is at most $n$ * $\frac{1}{2 * n}$ = $\frac{1}{2}$. So we prove SAT $\in$ RP. so NP $\subseteq$ RP. And we know that RP $\subseteq$ NP, so NP = RP.
\part{}
First, it is easy to show Oracle \#P is more powerful than Oracle PP, because if we know the count of all accepting path for a NTM, it is easy to calculate $Pr[D(x,y)]$ and is easy to judge whether to accept or reject the input and PP class can just tell the most significant bit in the \#P problem answer. So $P^{PP}$ $\subseteq$ $P^{\#P}$, now we will prove another side, $P^{\#P}$ $\subseteq$ $P^{PP}$.  \\
We want to show if we have a $PP$ oracle, we can solve $\#P$ problem using a poly-time algorithm using a $PP$ oracle. $PP$ just makes a decision and tells whether accepting path number is larger than rejecting path number while $\#P$ tells the conceptual number. We can use a binary search method to get the number calculated by $\#P$, because the number $m$, 0 $\leq$ $m$ $\leq$ $2^{n}$, we can do a binary search, the iteration time is just O(log($2^{n}$)) = O(n), we set $l$ to be $-2^{n}$, $r$ to be $-2^{n}$. In the loop, we just set a $val$ = $\frac{l+r}{2}$, if $val$ is larger than 0, we add $val$ rejecting paths, otherwise, we add $val$ accepting paths, and then we use the $PP$ oracle to tell if the problem belongs to $PP$, if so, we set $l$ to be $val + 1$, otherwise we set $r$ to be $val$. If $l$ $geq$ $r$, we jump the loop. After the loop, we can take $\frac{2^{n} + r}{2}$ to be the answer. Now that we can solve $\#P$, so $P^{\#P}$ = $P^{PP}$.
\part{} 
Because we know $\xi_{2}$ is larger than $\xi_{1}$, we just use Chernoff bound for the middle point $\frac{\xi_{1} + \xi_{2}}{2}$.
Our randomized algorithm runs as below: \\
We will run a random turning machine for D for $n$ times, and compare the total sum of the results to $\frac{\xi_{1} + \xi_{2}}{2}$ * $n$. if the total sum is larger than it, we accept, otherwise we will reject. We will use two useful forms of Chernoff bound to get the upper bound of the probability we output the wrong result. \\
Given $n$ independent random varibles, $x_{1}$..$x_{n}$ between [0, 1], let $X$ to be the sum of them and $u$ to be the expected value of $X$. \\
Pr(X $\leq$ (1 - $\sigma$) * $u$) $\leq$ $e^{\frac{-\sigma^{2}*u}{2}}$ \\
Pr(X $\geq$ (1 + $\sigma$) * $u$) $\leq$ $e^{\frac{-\sigma^{2}*u}{3}}$ \\
These two inequalities tell us about the upper bound when $X$ deviates from $u$ to some extent. \\
Given input $x$, if $x$ belongs to $L$, we know $u$ $geq$ $n$ * $\xi_{2}$ when we run the algorithm for $n$ times, and what is the possibility when we make a mistake that means, $X$ is less than $\frac{\xi_{1} + \xi_{2}}{2}$ * $n$. We apply the first Chernoff bound and set $\sigma$ to be $\frac{\xi_{2} - \xi_{1}}{2*\xi_{2}}$. (from this we know $\sigma$ is constant and will not change with the change of $n$). In the same way, we can set $\sigma$ to be $\frac{\xi_{2} - \xi_{1}}{2*\xi_{1}}$ in the second Chernoff bound. Thus, we know we can run the randomized algorithm for $n$ times, ($n$ is got by making $e^{\frac{-\sigma^{2}*u}{2}}$ and $e^{\frac{-\sigma^{2}*u}{3}}$ to be less than $\frac{1}{3}$), $n$ is poly-size compared to input, so we proved that $L$ $\in$ BPP.
\part{}
Assume SAT belongs to P/log(n). We will begin by using the self-reducibility of SAT. We give an algorithm for SATPath problem which helps finding a path for a circuit, returning None if the circuit is not satisfiable.\\
First if $\varphi$ $\notin$ SAT, returns None. \\
For i in (1,n): \\
	we just check if $\varphi$ with $x_{1}..x_{i-1}$ set and $x_{i}$ set to be 0 $\in$ SAT. If no, set $x_{i}$ = 1. \\
Return the assignment for $x_{1}..x_{i}$. \\
From this algorithm, we can find the SATPath problem $\in$ P/log(n), then we will give a poly-nomial algorithm for SAT, we just loop every possible advice string, (just O(n) possibilities), and run the TM for SATPath, we check each answer assignment (if the result is not none), if the assigment trully satisfies the formula, accept. After loop for all possible advice string, if we accept no result, we reject.
\part{}
Assume we know PSPACE $\subseteq$ P/poly, and because P/poly is the same as a series of poly-size circuits, we will prove a PSPACE-Complete problem, TQBF $\in$ $\Sigma_{2}$. Given a TQBF formula with size $m$, we know there is a series of poly-size circuits, $C_{1}..C_{m}$ to solves TQBF game with less than size $m$. So what we will do is to guess these circuits and check the correctness of these circuits. The algorithm in $\Sigma_{2}$ is below:
Given a TQBF formula $\Phi$ with m variables. \\
For the $\exists$ statement, we randomly guess a series of poly-size circuits first, $C_{1}..C_{m}$. \\
For the $\forall$ statement, we randomly check a sub-TQBF game $\Phi^{'}$ with $x_{1}..x_{i}$ set, if the quantifier is an existential quantifier, we just check if the output of $C_{k}$ for $\Phi^{'}$ is 
the output of $C_{k-1}$ for $\Phi^{''}$ with $x_{i+1}$ set to be 1 || the output of $C_{k-1}$ for $\Phi^{'''}$ with $x_{i+1}$ set to be 0. If the quantifier is an universal quantifier, we just check if the output of $C_{k}$ for $\Phi^{'}$ is the output of $C_{k-1}$ for $\Phi^{''}$ with $x_{i+1}$ set to be 1 \&\& the output of $C_{k-1}$ for $\Phi^{'''}$ with $x_{i+1}$ set to be 0. \\
So we have proved that Given PSPACE $\subseteq$ P/poly, PSPACE = $\Sigma_{2}$
\part{}
We will step by step solve the matrix-permanent problem using the \#SAT oracle in poly-time algorithm. \\
First, we consider a graph weight cycle-cover problem, and non-negative matrix-permanent problem, we can easily find the weight cycle-cover of the adjacent matrix constructed by the matrix is equal to the permanent of the matrix. \\
Then we want to prove the 0-1 permanent problem belongs to $\#P$. It is easy to see. Given a matrix $A$ with size $n$ * $n$, we just nondeterministicly select a permutation $k_{1}..k_{n}$ and our verification TM in $\#P$ is described below: \\
for i in $1..n$, if $A[i][k_{i}]$ = 0: reject. \\
After the loop, we accept. \\
It is easy to see the number of paths accepted by this TM equals the permanent of the given matrix, so 0-1 permanent problem belongs to $\#P$, we can reduce it to $\#SAT$, and uses a SAT oracle to solve it. \\
Then we will make a reduction from the non-negative matrix-permanent problem to 0-1 permanent problem. \\ 
\includegraphics[scale=0.1]{part1} \\
From the picture, we assume in the graph, there is an edge with weight, 7, because 7 = $2^{2}+2^{1}+2^{0}$, we split this edge into 9 edges and 3 nodes, it is easy to see the cycle-cover weights of two graphs are same(because we combine three paths in the new graphs, we can get 7). Then we split edges with weights $2^{r}$ into $2*r$ nodes and $4*r$ edges, and according to the Multiplication Principle, we can easily see the cycle-cover weight remains unchange in the new graph, so we finally get a 0-1 weight graph with $O(e*log(m))$ nodes, where $e$ is the number of edges in the original graphs, $m$ is the largest weight. (So we get a poly-size new graph). \\
Finally, we will reduce a general matrix-permanent problem to a non-negative matrix-permanent problem. Givn a matrix $A$ with size $n$ * $n$, we makes $m$ to be the largest magnitude. \\
It is easy to see |Perm(A)| $\leq$ $n!*m^{n}$
We give a number $L$ = $2*n!*m^{n} + 1$, we get a new matrix $A_{1}$ = $A$ mod $L$, and comput $P_{1}$ = Perm($A_{1}$) mod $L$. \\ 
If $P_{1}$ < $\frac{L}{2}$, we set Perm(A) = $P_{1}$, otherwise, we set Perm(A) = $P_{1}$ - $L$. 
(For general reduce, referred in: wikipedia).
So, in general, we can reduce a matrix-permanent problem to a non-negative matrix-permanent problem, and then reduce to a 0-1 matrix-permanent problem and then reduce it to $\#SAT$, and solve it by $\#SAT$ oracle, all the reduce is in poly-time and generate poly-size new problem compared to $n$, so we solve the matrix-permanent problem in poly-time with access to $\#SAT$ oracle.

\part{}
Assume size of S is $n$. \\
From the problem, we know H(s) != t means there is a t for every s, H(s) != t. First, let assume $n_{t}$ to be the number of values in $S$ that make H(s) = t. So First let's calculate Pr($n_{t}$ = 0) given a $t$. First, because $h$ is randomizedly chosen from the universal hash family. So E($n_{t}$) = $\frac{n}{2^{m}}$. Then let's calculate Var($n_{t}$). From the definition, Var($n_{t}$) = E($n_{t}*n_{t}$) - $E(n_{t})^{2}$. Let assume $x_{i}$ = 1 if H($s_{i}$) = t, otherwise $x_{i}$ = 0. \\
So E($n_{t}*n_{t}$) = E($(x_{1} + x_{2} + ... + x_{n})^{2}$), and because universal hash family (as taught in the class) is pairwise-independent, E($(x_{1} + x{2} + ... + x_{n})^{2}$) = n * $\frac{1}{2^{m}}$ + n * (n - 1) * $\frac{1}{2^{2*m}}$. So Var($n_{t}$) = n * $\frac{1}{2^{m}}$ + n * (n - 1) * $\frac{1}{2^{2*m}}$ - $\frac{n * n}{2^{2 * m}}$ = n * ($\frac{1}{2^{m}}$ - $\frac{1}{2^(2*m)}$). \\ 
So we can apply the ChebyShev Inequality, Pr($n_{t}$ = 0) $\leq$ Pr(|$n_{t}$ - $\frac{n}{2^{m}}$| $\geq$ $\frac{n}{2^{m}}$) $\leq$ = $\frac{n * (\frac{1}{2^{m}} - \frac{1}{2^{2*m}})}{\frac{n^{2}}{2^{2*m}}}$ $\leq$ $\frac{2^{m}}{n}$. \\

So Pr(H(S) != t) = Pr($n_{t_{1}}$ = 0 | $n_{t_{2}}$ = 0 |...| $n_{t_{2^{m}}}$ = 0) $\leq$ $2^{m}$ * $\frac{2^{m}}{n}$ = $\frac{2^{2*m}}{n}$ $\leq$ $\frac{2^{2*m+1}}{n}$ 
\part{}
We can use the similar approach to give a randomized algorithm to approximate \#$\phi$ as told in class. Assume $2^{k}$ $\leq$ |S| $\leq$ $2^{k+1}$. We just want to get the $k$ to use $2^{k+1}$ to approximate the answer. We consider a problem, given a set $T$ $\{0,1\}^{m}$, we randomly select $r$ hash function $h_{i}$ from the universal hash family. Consider a decision problem, if one of these hash functions make h(S) = T. We can draw a chart, size of $T$ from 0 to $m$. It is easy to show, when $m$ is too small, we use the inequality of question 2, when 
$\frac{2^{2m+1}}{|S|}$ is less than $\frac{1}{2}$, use a larger $r$ almost the decision problem provided will return false, and as $r$ get larger, the decision problem provided will return true for some probability because we randomly select hash function $h_{1}...h_{r}$, and we can similarly use SAT-Oracle to solve this decision problem, and consider the use this decision problem, if $m$ is set to be $\geq$ $k+1$, we can not get:  h(S) = T, and use $2*m+1$ $\leq$ $k-1$, we can know the probability Pr(H(S) = T) $\geq$ 1 - $\frac{2^{2m+1}}{|S|}$ $\geq$ $\frac{1}{2}$. So pr that one of the hash function satisfiy H(S) = T is $\geq$ 1 - $\frac{1}{2^{r}}$, almost 1. \\
By far we can know if $m$ less than $\frac{k}{2}$, decision problem will always output yes, is $m$ is larger than $k$, the decision problem will always output no, the undefine region is between $k/2$ to $k$, so we use the first $m$ that outputs yes, and use $M^{*}$ = $2^{m}$ to be the answer. we know $\frac{M^{*}}{2^(k/2)}$ $\leq$ \#$\phi$ $\leq$ $M^{*}$, and we will apply the similar tricky approach to this boundary, we just put n $\phi$ with length n together to get a $\phi^{'}$ with length $n^{2}$, and get the approach $M^{*}$, then use $\sqrt[n]{M^{*}}$ to be the approximating answer. And we consider the divisor will be $\sqrt[n]{2^{k/2}}$ $\leq$ 2, so we get the right approximating answer.
\end{document} 