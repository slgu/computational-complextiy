\documentclass[11pt]{article}
\usepackage{enumerate}
\usepackage[T1]{fontenc}
\usepackage[top=1in, bottom=1in, left=1.25in, right=1.25in]{geometry}
\newcommand{\itab}[1]{\hspace{0em}\rlap{#1}}
\newcommand{\tab}[1]{\hspace{1em}\rlap{#1}}
\title{Homework 2 for Intro to Computational Complextiy}
\author{Shenlong Gu}
\date{917-544-8927, sg3301@columbia.edu, 27, September 2015}
\begin{document}
\maketitle
\part{}
\def \vf {$V_{1}(x,c)$}
\def \vs {$V_{2}(x,c)$}
\def \v {$V(x,c)$}
\def \mf {$M_{1}$}
\def \ms {$M_{2}$}
\def \lf {$L_{1}$}
\def \ls {$L_{2}$}
    We assumes a Non-DTM \mf{} decides the language \lf, and 
    a Non-DTM \mf{} decides the language \ls. \\
    Let \vf{} be the verifier for \mf{} and \vs{} be the verifier
    for \ms{}.
\begin{enumerate}
\item {\bf union} \\
    We will construct a poly-time \v{} for the language, the union of 
    \lf{} and \ls{} to solve this problem.\\
    for any input $x$, given a stratage input string $c$, let \v{} = 1 iff \vf{} = 1 and \vs{} = 1. \\
    It is easy to see \v{} is the verifier which runs in polynomial time 
    given input $x$ and $c$ and \v{} is the verifier for the language, the
    union of \lf{} and \ls{}. So NP is closes under union operation. 
\item {\bf concatenation} \\
    We will construct a poly-time \v{} for the language, the union of 
    \lf{} and \ls{} to solve this problem.\\
    for any input $x$, the stratage input $c$ which is the union-encoding of the stratage $c_{1}$ for \vf{}, the stratage $c_{2}$ for \vs{}. \\
    \v{} works as the follows, creates $n + 1$ pairs, $(i, n - i), 0 \leq{} i \leq n$. Decode $c$ into $c_{1}$ and $c_{2}$ \\
    foreach pair $(i, n - i)$: \\
    \itab{} \tab{if {$V_{1}(x[0:i],c_{1})$} == 1 and {$V_{2}(x[i:n],c_{2})$} == 1:}\\
    \itab{} \tab{} \tab{\v{} = 1, accepts} \\
    end
    \v{} = 0, rejects. (Because no pairs work given the stratage $c_{1}$ 
    and $c_{2}$. \\
    It is easy to see \v{} is the verifier which runs in polynomial time 
    given input $x$ and $c$ and \v{} is the verifier for the language, concatenation of \lf{} and \ls{}. So NP is closes under concatenation operation. 
\item {\bf star} \\
    We will construct a poly-time \v{} for the language, the union of 
    \lf{} and \ls{} to solve this problem.\\
    Given an input $x$, we let $c$ to encode a segmentation of input $x$ into 
    $x_{1},x_{2},...x_{k}$, and a stratage of \vf{} to these $k$ segmentation $c_{1},c_{2},...c_{k}$. \\
    Then, \v{} works as follows: \\
    decode $c$ first, if $c$ can not be decoded, rejects it. \\
    for $i = 0$ to $k$: \\
    \itab{} \tab{if $V_{1}(x_{i},c_{i})$ == 1:} \\
    \itab{} \tab{} \tab{\v{} = 1, accepts} \\
    end \\
    \v{} = 0, rejects.\\
    It is easy to see \v{} is the verifier which runs in polynomial time 
    given input $x$ and $c$ and \v{} is the verifier for the language, star of \lf{}. So NP is closes under star operation. 
\end{enumerate} 
Then to prove $G$ and $H$ isomorphic. Given an input $x$ with length $n$, we encode a permutation of $n$ which is a node mapping from $G$ and $H$, 
we give a verifier \v{} given the input $x$ and encoding mapping $c$, it is easy to judge if two maps are isomorphic under this mapping in poly-time 
(just check if each mapping edge exists or doesn't exist in two graphs).
\part{}
    First, we will show that if a graph is bipartite, then it can not have a cycle containing an odd number of nodes. 
    Assume there is a such cycle the bipartite, $n_{1}$, $n_{2}$,.....$n_{2*k+1}$. We color the nodes into two part(blue and red)
    . if $n_{1}$ is red, $n_{2}$ is blue, then $n_{2*k+1}$ will be red the same as $n_{1}$, and there is an edge between $n_{1}$ and $n_{2*k+1}$,
    so there comes a contradictory. \\ \\
    Second, if there is no such cycle, we will show it is a bipartite. We also color the node, we randomly selects a node, and gives a color (for example, red),
    We traverse the edge from this node using dfs or bfs, everytime there is an edge between two nodes, we color the child node with a opposite color from its parent, when there is a back edge. Because there is no cycyle with an odd number of nodes, this node will not result in a contradictory with its ancestor, so we can color all nodes without contradictory, so it is a bipartite. \\ \\
    Third, we will show bipartite problem $\in$ NL. \\ 
    We change the bipartite judging problem into a existing problem. As proved above, we can change a bipartite judging problem by deciding that there is no cycle containing odd number nodes.\\
    Then we think about the complement problem(because NL equals CO-NL): how to decide that if there is a cycle containing odd number nodes. \\
    To use NDTM to solve this problem: \\
    First, we nondeterministicly selects a start node, $s$, and set a next node $t$ to nondeterministicly be a child of $s$, and set a counter to be zero .\\
    while counter $\leq$ $n$:\\
    ($n$ is total number of nodes, because the length of the cycle can not exceed $n$) \\
    \itab{} \tab{if $t$ equals $s$:} \\
    \tab{} \tab{} \tab{if counter is odd:}\\
    \tab{} \tab{} \tab{} \tab{accepts it}\\
    \tab{} \tab{} \tab{else:}\\
    \tab{} \tab{} \tab{} \tab{rejects it}\\
    \itab{} \tab{We nondeterministicly set $t$ to be a child of $t$,(if there is no child, we rejects)}\\
    \itab{} \tab{counter = counter + 1} \\ 
    We see that this NDTM decides the problem (odd-number-nodes cycle existing problem), and the space it uses is just start node number, next node number, counter which cost just logarithmic space. So bipartite judging problem $\in$ NL. 
\end{document}
