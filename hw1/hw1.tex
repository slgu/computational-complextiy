\documentclass[11pt]{article}
\usepackage{enumerate}
\usepackage[T1]{fontenc}
\newcommand{\itab}[1]{\hspace{0em}\rlap{#1}}
\newcommand{\tab}[1]{\hspace{1em}\rlap{#1}}
\title{Homework 1st for Computational Complextiy}
\author{Shenlong Gu}
\date{sg3301@columbia.edu, 14, September 2015}
\begin{document}
\maketitle
\part{}
We make some assumptions, turning machine M1 decides language L and turning machine M2 decides language L' \\
\def \mf {$M_{1}$}
\def \ms {$M_{2}$}
\def \mt {$M^{'}$}
\def \Lt {$L^{'}$}
\begin{enumerate}[a)]
\item  {\bf union} \\
        We give a turning machine \mt, which works as follows. \\
        For a given string $l$,\\ 
        Simulate $l$ on \mf{}, \\ 
        if \mf{} accepts $l$: \\ 
        \itab{} \tab{\mt{} accepts $l$.} \\ 
        else: \\
        \itab{} \tab{simulate $l$ on \ms{}.} \\
        \itab{} \tab{if \ms{} accepts $l$:} \\
        \itab{} \tab{} \tab{\mt{} accepts $l$.} \\
        \itab{} \tab{else} \\
        \itab{} \tab{} \tab{\mt{} rejects $l$.} \\ 
        We can find \mt{} decides the language union of $L$ and $L^{'}$, so union operation is closed for two decidable languages.

\item {\bf concatenation} \\ 
        We give a turning machine \mt{}, which works as follows. \\
		For a give string $l$ with length n. \\
		We enumerates $i$ from 0 -> $n$ to generate $n + 1$ pairs $(i, n - i)$ \\
        For each pair$(i, n - i)$:\\ 
        \itab{} \tab{we simulate $l[0:i]$ on machine \mf{} and $l[i:n]$ on machine \ms{}} \\
        \itab{} \tab{if \mf{} accepts $l[0:i]$ and \ms{} accepts $l[i:n]$} \\
        \itab{} \tab{} \tab{\mt{} accepts $l$} \\
        \itab{} \tab{else} \\
        \itab{} \tab{} \tab{continue} \\
        End \\
        \mt{} rejects $l$. (because after enumerating each pair, we can not find a solution) \\ 
        We can find \mt{} decides the language concatenation of $L$ and \Lt{}, so concatenation operation is closed for two decidable languages. 

\item {\bf complementation} \\ 
        We give a turning machine \mt{}, which works as follows: \\
        For a given string $l$, \\ 
        simulate l on \mf{}.\\
        if \mf{} accepts $l$:\\
        \itab{} \tab{\mt{} rejects $l$.}\\
        else: \\
        \itab{} \tab{M' accepts $l$.} 
\item {\bf intersection} \\
        We give a turning machine \mt{}, which works as follows: \\
		for a given string $l$, \\ 
        simulate $l$ on \mf{}, \\ 
        if \mf{} rejects $l$: \\
        \itab{} \tab{\mt{} rejects $l$.} \\
		else: \\
        \itab{} \tab{simulate $l$ on \ms{},}\\
        \itab{} \tab{if \ms{} rejects $l$:} \\
        \itab{} \tab{} \tab{\mt{} rejects $l$.} \\
        \itab{} \tab{else:} \\ 
        \itab{} \tab{} \tab{\mt{} accepts $l$.} \\
        We can find \mt{} decides the language intersection of $L$ and \Lt{}, and intersection operation is closed for two decidable languages.
\end{enumerate}
\part{}
\end{document}
